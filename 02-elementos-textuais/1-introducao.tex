\chapter{Introdução}

\section{Contextualização e Motivação}
No Brasil, graças a sua extensão e situação geográfica, são cultivadas todas as fruteiras dos climas tropical, sub-tropical e temperado. Este fato permite o cultivo de fruteiras de grande valor econômico, como bananeiras, laranjeiras, entre outras. Em consequência disso, o país é um dos maiores produtores de frutas do mundo.

Os fatores climáticos podem ser os grandes vilões na produção de mudas. Intempéries como temporais, geadas, altas temperaturas ou até mesmo grandes períodos de secas podem interferir de forma maléfica no desenvolvimento de plantas, trazendo grandes transtornos ao produtor. A fim de se contornar essas adversidades, uma alternativa  tem sido o cultivo em ambientes protegidos.

\begin{citacao}
    O cultivo protegido consiste em uma técnica que possibilita certo controle de variáveis climáticas como temperatura, umidade do ar, radiação solar e vento. Esse controle se traduz em ganho de eficiência produtiva, além do que o cultivo protegido reduz o efeito da sazonalidade, favorecendo a oferta mais equilibrada ao longo dos meses \cite{silva2014cultivoprotegido}.	
\end{citacao}

Existem diversos tipos de ambientes protegidos como túneis e ripados, sendo que o mais comum dentre eles é a estufa. Segundo 
\citeonline{nakamae1999agrianual}, no Brasil a área de cultivo sob plástico era de cerca de 2 mil hectares, sendo que apenas no estado de São Paulo havia 897 hectares de estufas. 

Segundo \citeonline{cermeno1990estufas}, o uso adequado desses ambientes pode proporcionar maior produtividade, se comparado ao cultivo realizado nos moldes tradicionais, chegando a ser de 2 a 3 vezes mais produtivo e com qualidade superior.

Nesses ambientes, o controle das condições climáticas pode ser feito por meio de diversos equipamentos tais como, ventiladores, nebulizadores, entre outros. Estes equipamentos podem ser controlados de forma manual ou por meio de sensores previamente programados.

No Instituto Federal de Educação, Ciência e Tecnologia do Sul de Minas Gerais (IFSULDEMINAS) campus Muzambinho, é feito a produção de mudas em ambiente protegido (estufa). Este ambiente possui um sistema de nebulizador responsável por reduzir a temperatura do ar e manter a umidade relativa alta, entretanto este sistema é acionado de forma totalmente estática, sem levar em consideração as condições reais à que se encontra o ambiente, o que pode comprometer o desenvolvimento e/ou qualidade da muda.

Levando em consideração o que foi exposto, este trabalho tem como objetivo aplicar uma rede de sensores sem fio para acompanhamento e controle das condições internas da estufa, a fim de se obter um ambiente com as condições ideais o que possibilitará maior produtividade e qualidade no desenvolvimento das plantas.

\section{Objetivos}

\subsection{Objetivo Geral}
Melhorar o controle do sistema nebulizador existente no setor de fruticultura do IFSULDEMINAS campus Muzambinho, aplicando uma Rede de Sensores sem fio (RSSF) para acompanhamento e controle das condições internas da estufa, a fim de se obter um ambiente com as condições ideais possibilitando maior produtividade e qualidade no desenvolvimento das plantas.

\subsection{Objetivos Específicos}
\begin{itemize}
    \item Desenvolver uma rede de sensores sem fio.
    \item Organizar os dados coletados em uma base de dados.
    \item Propor uma aplicação WEB para apresentação dos dados ao usuário empregando conceitos de usabilidade.
    \item Contribuir com o setor de Fruticultura do IFSULDEMINAS Campus Muzambinho por meio do desenvolvimento deste trabalho.
    \item Fomentar a integração do curso de Ciência da Computação com as outras áreas presentes no campus.
\end{itemize}

\section{Estrutura ou Organização do Trabalho}
Este trabalho está dividido em capítulos. O capítulo 2 (Revisão de Literatura) traz uma revisão dos principais conceitos envolvidos no desenvolvimento deste trabalho. O capítulo 3 (Metodologia) apresenta os materiais e o métodos que foram empregados no desenvolvimento. No capítulo 4 (Resultados e Discussão) são apresentados os resultados alcançados ao longo do desenvolvimento deste trabalho. O capítulo 5 (Considerações Finais) traz as conclusões obtidas com este trabalho além de possíveis trabalhos futuros a serem desenvolvidos. Por fim o capítulo 6 (Referências) apresenta todas as referências que foram utilizadas para embasar este trabalho.