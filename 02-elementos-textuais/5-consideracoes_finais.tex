\chapter{Considerações Finais}

\section{Rede de Sensores}
Com a rede de sensores desenvolvida pode-se concluir que houve melhorias no controle do sistema nebulizador, devido ao fato de durante o período da noite os níveis de umidade relativa do ar se manterem bem altos e constantes, nestas condições não há a necessidade de estar acionando o sistema de nebulização. 

Da forma como era controlado anteriormente o sistema de nebulização era acionado em intervalos de tempo programados. Agora com a rede de sensores desenvolvida, o sistema só é acionado se as condições climáticas da estufa não atenderem aos parâmetros definidos pelo usuário.

\section{Aplicação WEB}
Conclui-se que o desenvolvimento da aplicação web foi de grande utilidade para o setor, o que possibilitou o acompanhamento das condições internas da estufa em tempo real de qualquer lugar e por meio de qualquer dispositivo. Além de que a aplicação foi desenvolvida pensando na facilidade de uso dos usuários, por meio de uma interface limpa e objetiva, o que contribui para que os usuários se adaptem ao sistema.	

A aplicação desenvolvida também possibilita ao usuário configurar os valores de máximo e mínimo para cada variável monitorada pela rede, desta forma esses valores podem ser alterados de forma dinâmica, sem ter que ficar alterando os valores direto no código do nós.

\section{Proposta de Trabalhos Futuros}
Fica como proposta de trabalhos futuros, a implementação da parte de gerenciamento de experimentos, para que o usuário possa manter todos os parâmetros que estão sendo aplicados no experimento e não somente as variáveis climáticas.

Em relação a rede de sensores, fica a sugestão de que sejam monitoradas mais variáveis como umidade e ph do solo e radiação solar.