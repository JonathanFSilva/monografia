%
% Documento: Resumo (Português)
%
\begin{RESUMO}

\thispagestyle{empty}
    \begin{SingleSpace}
	\noindent
		{\expandafter\uppercase\expandafter{\imprimirsobrenomeautor}}, {\imprimirnomeautor}. \textbf{{\imprimirtitulotb}}. {\imprimirdata}. \pageref{LastPage}f. Trabalho de Conclusão de Curso (Curso {\imprimirprograma}) – {\imprimirinstituicao} – {\imprimirdepartamento}, {\imprimirlocal}, {\imprimirdata}.
		
	\end{SingleSpace}
    
    \vspace{1cm}
    
    \begin{center}
	    \textbf{RESUMO}
    \end{center}
    
	\begin{SingleSpace}
	\hspace{-1.2 cm}O clima é fundamental para o desenvolvimento de plantas, os fatores climáticos como  temperatura, umidade e luminosidade podem interferir de forma benéfica ou maléfica no desenvolvimento da planta, sendo assim, controlar esses fatores é de suma importância e o uso do ambiente protegido vem somar a essa busca por melhores resultados. Em ambientes controlados, como estufas, as condições ambientais pode ser alteradas por meio de vários equipamentos como aquecedores, nebulizadores, ventiladores, entre outros. Esses equipamentos podem ser controlados manualmente ou por meio de sensores previamente programados. 
	O objetivo deste trabalho é mostrar como foi desenvolvida uma rede de sensores sem fio que monitora as variáveis microclimáticas em uma estufa onde é realizado o cultivo de mudas por estaquia. A rede coleta dados como temperatura, umidade relativa do ar e molhamento foliar, que são exibidos em uma aplicação web por meio de gráficos e relatórios, além de também serem utilizados como parâmetros no controle do sistema de nebulização presente na estufa. 
	
	\vspace*{0.5cm}\hspace{-1.3 cm}\textbf{Palavras-chave}: Rede de Sensores sem fio. Fruticultura. Automação. Monitoramento. Estufa.
		
	\end{SingleSpace}
\end{RESUMO}


