%
% Documento: Resumo (Português)
%
\begin{RESUMO}

\thispagestyle{empty}
    \begin{SingleSpace}
	\noindent
		{\expandafter\uppercase\expandafter{\imprimirsobrenomeautor}}, {\imprimirnomeautor}. \textbf{{\imprimirtitulotb}}. {\imprimirdata}. \pageref{LastPage}f. Trabalho de Conclusão de Curso (Curso {\imprimirprograma}) – {\imprimirinstituicao} – {\imprimirdepartamento}, {\imprimirlocal}, {\imprimirdata}.
		
	\end{SingleSpace}
    
    \vspace{1cm}
    
    \begin{center}
	    \textbf{RESUMO}
    \end{center}
    
	\begin{SingleSpace}
	\hspace{-1.2 cm}O clima é fundamental para o desenvolvimento de plantas, os fatores climáticos como  temperatura, umidade e luminosidade podem interferir de forma benéfica ou maléfica no desenvolvimento da planta, sendo assim, controlar esses fatores é de suma importância e o uso do ambiente protegido vem somar a essa busca por melhores resultados. Em ambientes controlados, como estufas, as condições ambientais pode ser alteradas por meio de vários equipamentos como aquecedores, nebulizadores, lâmpadas, tela escura, outros. Esses equipamentos podem ser controlados manualmente ou por sensores que ativam os vários equipamentos (previamente programados) responsáveis pelo controle do ambiente. 
	Este trabalho tem como objetivo o uso do conceito de Internet das Coisas como ferramenta para o desenvolvimento de uma rede de sensores sem fio a fim de efetuar o monitoramento de variáveis microclimáticas no cultivo de mudas. Inicialmente será implementado uma rede de sensores sem fio, a fim de efetuar a coleta de dados como temperatura, umidade, luminosidade e molhamento foliar, juntamente com um sistema web para possibilitar o acompanhamento e melhor visualização desses dados. A partir desses dados serão tomadas decisões, como por exemplo, o acionamento de uma válvula de irrigação.

	\vspace*{0.5cm}\hspace{-1.3 cm}\textbf{Palavras-chave}: Rede de Sensores sem fio. Fruticultura. Automação. Monitoramento. Estaquia.
		
	\end{SingleSpace}
\end{RESUMO}


