%
% Documento: Resumo (Inglês)
%

\begin{ABSTRACT}

    \begin{SingleSpace}
	   \noindent
		{\expandafter\uppercase\expandafter{\imprimirsobrenomeautor}}, {\imprimirnomeautor}. \textbf{{\imprimirtitulotb}}. {\imprimirdata}. \pageref{LastPage}f. Trabalho de Conclusão de Curso (Curso {\imprimirprograma}) – {\imprimirinstituicao} – {\imprimirdepartamento}, {\imprimirlocal}, {\imprimirdata}.
		
	\end{SingleSpace}
    
    \vspace{1cm}
    
    \begin{center}
	    \textbf{ABSTRACT}
    \end{center}%Alinhamento centralizado
    
	\begin{SingleSpace}
	   % Abstract
		\hspace{-1.3 cm}The climate is fundamental for the development of plants, the climatic factors such as temperature, humidity and luminosity can interfere in a beneficial or evil way in the development of the plant, therefore, controlling these factors is of paramount importance and the use of the protected environment adds up to this search for better results. In controlled environments, such as greenhouses, environmental conditions can be altered by means of various equipment such as heaters, nebulizers, lamps, dark screen, others. These devices can be controlled manually or by sensors that activate the various equipment (previously programmed) responsible for the control of the environment.
		This work aims to use the Internet of Things concept as a tool for the development of a wireless sensor network in order to monitor microclimatic variables in seedling cultivation. Initially a network of wireless sensors will be implemented in order to collect data such as temperature, humidity, lightness and leaf wetness, together with a web system to enable monitoring and better visualization of these data. From these data decisions will be made, such as the activation of an irrigation valve.

		\vspace*{0.5cm}\hspace{-1.3 cm}\textbf{Keywords}: Wireless Sensor Network. Fruticulture. Automation. Monitoring. Staking.

		
	\end{SingleSpace}

\end{ABSTRACT}
