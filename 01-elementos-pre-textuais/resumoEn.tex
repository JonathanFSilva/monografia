%
% Documento: Resumo (Inglês)
%

\begin{ABSTRACT}

    \begin{SingleSpace}
	   \noindent
		{\expandafter\uppercase\expandafter{\imprimirsobrenomeautor}}, {\imprimirnomeautor}. \textbf{{\imprimirtitulotb}}. {\imprimirdata}. \pageref{LastPage}f. Trabalho de Conclusão de Curso (Curso {\imprimirprograma}) – {\imprimirinstituicao} – {\imprimirdepartamento}, {\imprimirlocal}, {\imprimirdata}.
		
	\end{SingleSpace}
    
    \vspace{1cm}
    
    \begin{center}
	    \textbf{ABSTRACT}
    \end{center}%Alinhamento centralizado
    
	\begin{SingleSpace}
	   % Abstract
		\hspace{-1.3 cm}The climate is fundamental for the development of plants, the climatic factors such as temperature, humidity and luminosity can interfere in a beneficial or evil way in the development of the plant, therefore, controlling these factors is of paramount importance and the use of the protected environment adds up to this search for better results. In controlled environments, such as greenhouses, environmental conditions can be altered by means of various equipment such as heaters, nebulizers, fans, among others. These devices can be controlled manually or by means of previously programmed sensors.
        The objective of this work is to show how a wireless sensor network was developed that monitors the microclimatic variables in a greenhouse where the cultivation of seedlings by cutting is carried out. The network collects data such as temperature, relative air humidity and leaf wetness, which are displayed in a web application through graphs and reports, as well as being used as parameters in the control of the misting system present in the greenhouse.

		\vspace*{0.5cm}\hspace{-1.3 cm}\textbf{Keywords}: Wireless Sensor Network. Fruticulture. Automation. Monitoring. Greenhouse.

		
	\end{SingleSpace}

\end{ABSTRACT}
